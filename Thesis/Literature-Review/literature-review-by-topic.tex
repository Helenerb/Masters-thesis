\newpage
\section{Mortality Modeling as a Statistical Field}
\label{sec:literature_review_by_topic}
\textcolor{myDarkGreen}{Her kan du snakke litt om historien til mortality modellering, og gjennomgå de viktigste modellene. Du kan nevne Lee-Carter modellen, og hvordan mange forskjellige versjoner av den har vært populær. Du kan nevne alternative modeller, og si litt om hva som har blitt sagt og gjort med disse tidligere - gjerne sett i forhold til Lee-Carter. Du kan også nevne forskjeller i Cause-specific og generell mortality. For eksempel at cohort har vært vanlig i cause-specific mortality, og nå også blir vanligere for general mortality. }
Good statistical methods for modeling trends in mortality rates are important tools in several scientific fields. In the public sector, projections of general and cause-specific mortality are of key interest in, among others, the development of health policies and planning of social securities and pension funds, and in actuarial applications, mortality forecasts play an important role in the pricing of life insurances and pension products (\textcite{LeeCarter1992}, \textcite{Brouhns_2002}, \textcite{Czado_2005}, \textcite{Renshaw_Haberman_2006},\textcite{Renshaw_Haberman_2009}). 

The dominating approach to mortality modelling on a populational level is to consider mortality, in the form of observed counts of mortality, mortality rates or probability of mortality, as a combination of age- and period specific effects. In some cases, a cohort effect is included, which describes lifelong effects for the population with the same birth year. This general structure is common for most mortality models in literature. The differences between the numerous models that we see in the field of mortality modeling lies in the structure of the relationship between the time effects, how variability is captured by the model, whether a separate cohort effect is included and whether the model is considered in a Bayesian setting. \textcolor{myDarkGreen}{Dette har du egentlig ikke kilder på. Burde du ha det?}

\subsection{The Dominating Models for Mortality Modeling}
\label{sec:most-common-mortality-models}
\textcolor{myDarkGreen}{In this section, you will account for the most popular mortality models. How are they structured, for what purpose are they mainly used. You do not need to talk about advantages/ disadvantages here. Just say what they are, how they are defined etc. Perhaps also mention common model choices for random effects. }

\textcolor{myDarkBlue}{The most basic mortality model is the APC model, which defines a linear, additive relationship between the age, period and cohort effects. This model has been used in medical and demographic sciences (see e.g. \textcite{Clayton1987} or \textcite{Hobcraft_1982}), and is often applied to cause-specific mortality \parencite{RieblerThesis2010}. Its simple structure makes it easy to perform inference with, especially when it is considered in a Bayesian setting. However, due to the linear dependence between the time effects, arising from the relation
\begin{equation*}
    \text{cohort} = \text{period} - \text{age},
\end{equation*}
this model is especially sensitive to identifiability issues between the three time effects. 
}

In the actuarial sciences, the model proposed by \textcite{LeeCarter1992}, hereby referred to as the Lee-Carter model, has been popular ever since its introduction, and variations if this model has largely dominated the field of mortality modeling for actuarial purposes (\textcite{Renshaw_Haberman_2009}, \textcite{booth_tickle_2008}). In its most basic form, the Lee-Carter model assumes a Gaussian distribution of the log-mortality rates, where the mean of the distribution is linked to some age- and period effects through a predictor:
\begin{equation}
    \log(m_{x,t}) = \alpha_x + \beta_x\kappa_t + \eps_{x,t}.
\end{equation}
Here $m_{x,t}$ is the central mortality rate for an age group $x$ at period $t$. $\eps_{x,t}$ is an error term with zero mean and some variance, representing the variability in the mortality that cannot be explained by any age- or period effects. $\alpha_x$ and $\kappa_t$ are the main age- and period effects, and $\beta_x$ is an age-specific modulating factor, that should reflect how sensitive the population at a given age is to period-specific changes in mortality. 

After its introduction, numerous extensions of the Lee-Carter model have been proposed. \textcite{Brouhns_2002} suggest to exchange the Gaussian log-mortality model structure with a Poisson structure, where they consider the observed number of deaths to follow a Poisson distribution:
\begin{equation*}
    Y_{x,t} \sim \Poisson(E_{x,t}e^{\eta_{x,t}}), \quad \eta_{x,t} = \alpha_x + \beta_x\kappa_t.
\end{equation*}
\textcite{Czado_2005} extended this model further, by incorporating it into a Bayesian framework. \textcite{Delwarde_2007} proposed extending the Poisson version of the Lee-Carter model with an error term to account for overdispersion, and \textcite{Wong_Forster_2018} considered this, together with an additional model structure accounting for overdispersion, in a Bayesian framework. 

Other extensions to the original Lee-Carter includes including multiple period terms, as suggested by \textcite{Renshaw_Haberman_2003}, to increase flexibility with respect to age. \textcite{Cairns_Blake_Dowd_2006} introduced a model directed at mortality at pensioner ages, suited for modeling of longevity risk, which is known as the CBD model. \textcolor{myDarkGreen}{Should you say something more about this?}

\subsubsection{Inclusion of a Cohort Term}
\label{sec:inclusion-of-cohort-term}
\textcolor{myDarkGreen}{This could be a subsection in its own - see how it is natural. But at least, talk about how some models include a cohort term and some does not. Talk about demographic significance? Talk about how the models are altered to inlclude the cohort term. }

\textcolor{myDarkBlue}{In the field of medical mortality modeling, the inclusion of the cohort effect have long been considered a useful tool to correctly capture mortality \cite{} ER DET FRA RIEBLER??} Additionally, in recent years, the need to include a explicit cohort effect also when describing tendencies in general mortality has been pointed out increasingly often in the actuarial community (e.g. \textcite{HUNT_Villegas_2015}). This is in contrast to more traditional Lee-Carter models, where cohort effects are acknowledged \textcolor{myDarkGreen}{Kilde eller stryk}, but treated as secondary to the age- and period effects \parencite{HUNT_Villegas_2015}.

The inclusion of this cohort effect can be formulated in several ways. The main division between the modeling of the cohort effect is whether it should be modulated with an additional age effect, in a similar manner to the period effect, or not. In the original proposition of the cohort-effect by \textcite{Renshaw_Haberman_2006} \textcolor{myDarkGreen}{Var det dem???}, it is age-modulated, but it has been suggested by e.g. \textcolor{myDarkGreen}{HVEM} that this only serves to further complicate the model and does not add additional insight/value \textcolor{myDarkGreen}{Du har det fra et sted, finn ut hvor}

Similarily, different choices of constraints have been considered for the cohort effect. \textcolor{myDarkGreen}{Vet ikke helt om denne skal her, men hvertfall snakk om man må ha constraints for "basic" identifiability, hvem som har brukt hvilke typer constraints, hvordan H&B har gjort det, siden de snakker mye om constraints}

\subsubsection{Multiplicative Term}
\label{sec:multiplicative-term}
\textcolor{myDarkGreen}{I don´t really think you need this as a subsection, you should just specifically mention it}
A common trait for most of the models mentioned above, is the inclusion of one or more the age-modulated period terms in the predictor. \textcolor{myDarkBlue}{This has been proven to be a good model choice, reflecting the data well. Additionally, as described more thoroughly in Section \ref{sec:identifiability_mortality_models}, for non-parametric model choices of the age-modulating term (usually $\beta_x$), the model is identifiable, as opposed to simpler linear structures, such as that of the APC model.}

\subsection{Choices of Constraints}
\label{sec:common-choices-of-constraints}
\textcolor{myDarkRed}{Perhaps join this paragraph with the Identifiability paragraph}

The mortality models are typically not identifiable in itself, and thus it is common practice to impose some constraints on the random effects to ensure identifiability.
\textcolor{myDarkBlue}{The constraints are arbitrary, and can in theory be any constraint that sufficiently reduces the degrees of freedom of the model. However, the demographic interpretation/significance of the random effect changes according to the choice of priors, and some constraints that ensure easy interpretation of the time effects are commonly seen throughout literature.} \textcolor{myDarkGreen}{Skriv om hva de forskjellige bruker, og hva slags betydning forskjellige constraints gir. }

In our experience, it is common practice for papers discussing models with modulating age effects to impose sum-to-unit constraints on these. It is also common to impose some identifying constraint on the period effect, however, the exact choice of this constraint varies in the literature, with the two most common choices being
\begin{equation*}
    \kappa_{t_1} = 0, \quad \sum_t\kappa_t = 0.
\end{equation*}

One common choice of constraints to let the first element of the period effect be at zero and to apply and sum-to-unit constraints on the modulating age effects:
\begin{equation}
    \kappa_0 = 0, \quad \sum_x\beta_x = 1.
    \label{eq:constr_first_kappa_zero}
\end{equation}
These constraints are used by e.g. \textcite{Renshaw_Haberman_2011}, .... \textcite{Renshaw_Haberman_2011} also applied, to the models including a distinct cohort effect, the constraint
\begin{equation}
    \beta_x > 0 \forall x,
\end{equation}
to be precautious. \textcolor{myDarkGreen}{Har ikke helt forstått hva de konkluderer med? Er det nyttig eller ikke?}

\textcite{NOTE: Vær sikker på at du faktisk trenger constraints på cohort-effekt for å ha identifiability!!! \textcite{Renshaw_Haberman_2011} ser ikke ut til å bruke det (men de bruker en two-step procedure for å fikse på forholdet mellom cohort, period og age... Se hva Andrea har gjort! Ganske sikker på at \textcite{Wisniowski2015} har brukt sum-to-zero constraints på cohort-effekten... }

\subsection{Mortality Models in a Bayesian Framework}
\label{sec:mortality_models_bayesian}
Her kan du snakke spesifikt om Bayesianske versjoner av Mortality modeller. Kanskje ikke her du skal argumentere for fordelene ved Bayesianske modeller, du kan evt referere til det hvis det står noe om det i annen litteratur - ja , det kan du! Snakk om utfordringer andre har hatt ved Bayesisk modellering - nevne convergence issues? Snakk om hvilke metoder som har vært brukt for å løse disse modellene.  

\textcolor{myDarkBlue}{The APC model has commonly been considered in a Bayesian framework. Its simple linear structure allows usage of the efficient \inla framework for Bayesian inference, which has an advantage in compulational speed over more traditional MCMC methods \parencite{rue2009inla}, which is typically the alternative for these kinds of models.}

\textcite{Czado_2005} proposed incorporating the Poisson-extended Lee-Carter model in a Bayesian framework, and they use MCMC methods to carry out the inference. 

An advantage with considering these models in a Bayesian framework, as pointed out by e.g. \textcite{Czado_2005}, is that incorporates uncertainty into the model and connected(?) predictions in a simple manner. \textcolor{myDarkBlue}{Since one obtains the model components as posterior marginal distributions as opposed to point estimates, finding the uncertainty of the estimates is straightforward.} Additionally, a Bayesian frameworks allows for easy incorporation of uncertainty about model parameters and previous expert knowledge, as argued by e.g. \textcite{Wisniowski2015}. 

\subsection{Common Model Choices of Time Effects}
\label{sec:common_choices_of_priors}
\textcolor{myDarkGreen}{The following is from the project thesis}
The age, period and cohort effects are also modeled differently throughout the literature, varying with different model assumptions and whether the model is considered in a Bayesian setting or not. 

One key difference is whether the effects, and the modulating age effect in particular, are modeled as parametric or non-parametric. \textcolor{myDarkBlue}{Hent fra H&B for eksempler på parametric. }

The modeling of the main age effect, $\alpha_x$ varies \textcolor{myDarkGreen}{Dette må du finne ut av}

The period effect $\kappa_t$ is usually modeled as some variation of a drifted time series model, typically a random walk with drift or an ARIMA model, as seen in e.g \textcite{Czado_2005}, \textcite{Brouhns_2002} and \textcite{LeeCarter1992}. This is a natural choice, as these time series models can be extended beyond the observed years, and by that produce forecasts.

For the remaining effects, there are less consensus on the model choices in the literature.

\textcolor{myDarkBlue}{A time-series approach is typically used to find forecasts of future mortality rates, and thus the period effect $\kappa_t$ is usually modelled as a random walk with drift (see for instance \textcite{LeeCarter1992}, \textcite{Wisniowski2015}, \textcite{Czado_2005}). The typical approach for mortality forecasts with the Lee-Carter model is then to fit the effects $\alpha_x$, $\beta_x$ and $\kappa_t$ to the available data, and use time-series models, such as the ARIMA model, to forecast future values of $\kappa_t$. Throughout literature, several different models have been used for $\alpha_x$, $\beta_x$ and $\gamma_k$. From previous literature, we see that $\alpha_x$ often takes the form of a smooth curve, increasing with $x$, while $\beta_x$ takes a more erratic shape. Both \textcite{Czado_2005} and \textcite{Wisniowski2015} have used a Gaussian distribution to model $\beta_x$, and \textcite{Wisniowski2015} have used Gaussian distribution to model $\alpha_x$ as well. Time series models have been used to model the cohort effect $\gamma_k$. Specifically, \textcite{Wisniowski2015} use an autoregressive process AR(1) to model $\gamma_k$ for mortality.}

\subsection{Identifiability in Mortality Models}
\label{sec:identifiability_mortality_models}
\textcolor{myDarkGreen}{Nevn resultatet til \textcite{hunt_blake_2020} som sier at dersom $\beta_x$ er non-parametric, så er cohort-effekten identifiserbar. Her kan du trekke inn den klassiske APC-modellen som eksempel på en modell der $\beta_x$ er parametrisk, og dermed også ikke-identifiserbar. Nevn også resultatet om at AP-modeller er definerte dersom man har gitte constraints. Nevn at H6B (oppdater kilde!!!) foreslår noen constraints på cohort-parameteren for å sikre identifiability, men at dette endrer demografisk signifikans og bør gjøres med omhu. }

\textcolor{myDarkBlue}{
\textcite{Renshaw_Haberman_2006} avoids the issues arising from the relation
\begin{equation*}
    \text{cohort} = \text{period} - \text{age}
\end{equation*}
by adopting the two-stage fitting procedure, first proposed by \textcite{LeeCarter1992}. This entails first establishing a static life table to represent the overall age effect $\alpha_x$, and then, conditioning on this age effect, establish the period and cohort effects, $\kappa_t$ and $\gamma_{t-x}$ and the modulating age effects $\beta_x$.}

\textcolor{myDarkGreen}{Discuss additional identifiability in models with unequal time intervals? I am not sure if simply updating the indexing scheme is sufficient. \textcite{RieblerThesis2010} suggests that these issues can be solved by applying smoothing functions to the effects, such as Rw2. You can discuss that we do this for the period effect, but we do not necessarily want to do it for the cohort effect, as we do not wish to encourage drift in the cohort term. \textcite{Holford_2006} discusses this issue quite thoroughly, you have not read this yet. What you can say is that one should treat cyclic patterns in the cohort effect with skepticism, as this may be a result of this identifiability issue. If you do not see any patterns like that, you can simply comment that. }

\subsection{Projection and Forecasting of Mortality}
\label{sec:forecasting_mortality}

\textcolor{myDarkGreen}{I am actually not quite sure if you need this as a separate paragraph...}

\textcolor{myDarkGreen}{
Hvilke metoder har andre brukt for å projektere (forecast - forsøk å forstå begrepene godt) mortality rates. Hvilke utfordringer er knyttet til forskjellige metoder?}

The classical approach when fitting and producing forecasts with Lee-Carter types of models is to first project the time effects, and then extend the time series model for the period effect to produce forecasts (see e.g. \textcite{LeeCarter1992}, \textcite{Renshaw_Haberman_2003}, \textcite{Brouhns_2002b}). Although widely used, this procedure have been criticized to produce possible incoherence \parencite{Czado_2005} and ... 

Several improvements of this have been proposed. \textcite{Czado_2005} argues that an advantage of the Bayesian approach is the possibility of incorporating estimation and forecasting in a single step.

When the cohort effect is included, a number of the earliest and latest cohorts is omitted from the data (see e.g. \textcite{Renshaw_Haberman_2009}). 

\subsection{Error Terms - Handling Uncertainty, Volality and Overdispersion}
\label{sec:error_terms}
\textcolor{myDarkGreen}{
Si noe om hvordan utviklingen har gått. Fra enkel error term, til Poisson-antagelse, til dobbel-usikkerhet - med både error term og Poisson. Hvilke papers bruker de forskjellige (hvis det gir mening, kan være det blir litt oppramsende)
}

\textcolor{myDarkGreen}{You just realised: The Gaussian error structure is homoscedastic, and the Poisson error structure is heteroscedastic? Read \textcite{Renshaw_Haberman_2003} to see where this is stated. Is this because the variance of the Gaussian structure is the same for all $x$, $t$? Since $\log(m_{x,t}) \sim \Normal(\eta_{x,t}, 1/\tau_\eps)$? So we expect the error to have the same variance across the field? But for the Poisson error structure, we expect the error to vary with the mean $\eta_{x,t}$. What is then the combination? Both heteroscedastic and homoscedastic? }

In the simplest Lee-Carter model, variability in the data that could not be attributed to age- or period effects will be assigned to a Gaussian error term. The popular Poisson-extension to the Lee-Carter model proposed by \textcite{Brouhns_2002}, the PLC model henceforth, accounts for variability in the data through the variability of the Poisson distribution, however, this approach limits the variability to be equal to the mean. This rigid structure of assumed homogeneity within each age-period cell in the PLC model is argued by many, e.g. \textcite{Wong_Forster_2018} to be undesirable. \textcite{Wong_Forster_2018} explicitly demonstrates that the PLC model fails to account for all variability in their example mortality data through residual plots showing overly large residuals. They attribute these inaccuracies to the inability of the PLC model to account for overdispersion, which they explain to be the variations in mortality within age-period cells caused by differences in smoking prevalence, income, genetic backgrounds etc. \textcite{Wong_Forster_2018} further argues that while some of these variations can be attributed to cohort effects across the data, not all variability can be explained in this way. 

One way to account for this overdispersion, which was proposed by \textcite{Delwarde_2007} and demonstrated to be efficient by \textcite{Wong_Forster_2018}, is to add an additional error term to the predictor in the PLC model. This error term can take many structures, and we refer here to the two versions of the PLC model extended with an error term discussed by \textcite{Wong_Forster_2018}. The first is what they call the PLNLC model, the Poisson log-normal Lee-Carter model:
\begin{equation*}
    \begin{aligned}
        Y_{x,t}\mid \mu_{x,t} \sim \Poisson(E_{x,t}\mu_{x,t}) \\
        \log(\mu_{x,t}) = \alpha_x + \beta_x\kappa_t + \nu_{x,t} \\
        \nu_{x,t}\mid \sigma_\mu^2 \sim \Normal(0, \sigma_\mu^2).
    \end{aligned}
\end{equation*}
The second model that they consider is the NBLC model, the negative binomial Lee-Carter model, first proposed by \textcite{Delwarde_2007}:
\begin{equation*}
    \begin{aligned}
        Y_{x,t}\mid \mu_{x,t} \sim \Poisson(E_{x,t}\mu_{x,t}) \\
        \log(\mu_{x,t}) = \alpha_x + \beta_x\kappa_t + \log(\nu_{x,t}) \\
         \nu_{x,t}\mid \phi \sim \gammaDist(\phi, \phi).
    \end{aligned}
\end{equation*}
This model structure was first proposed by \textcite{Delwarde_2007}. \textcite{Wong_Forster_2018} compares these two models to PLC, and their results show that the PLNLC model and the NBLC model gives very similar performance in goodness of fit, and that both perform significantly better than the PLC model. They recommend the NBLC model because of computational advantages. 

\subsection{Convergence in Mortality Models}
\label{sec:convergence_in_mortality_models}
\textcolor{myDarkBlue}{Throughout literature on mortality modeling, the ability of the different models and estimations procedure to converge, as well as their robustness to changes in initial values and changes in data, is a common topic of discussion. \textcite{HUNT_Villegas_2015} is dedicated to discussing these issues in depth, comparing both model structures and estimation procedures to explore how they give different results in goodness of fit, sensitivity to changes in initial values in the estimation procedures and robustness to changes in the data. Their paper is a response to the issues already raised by \textcite{Renshaw_Haberman_2009} and \textcite{Cairns_2009}, which discussed the advantages and potential issues with a one-stage estimation procedure compared to the traditional two-stage estimation procedure. }
\textcolor{myDarkGreen}{Si mer om hvordan man har lurt på om dette skyldes identifiability issues, om en Bayesiansk approach kan løse det, om ekstra constraints kan løse det, sånn som \textcite{HUNT_Villegas_2015} foreslår, om noe av det bare er at man ikke kan finne unike løsninger i dataen. }

Her kan du diskutere det som har blitt sagt om konvergensproblemer for dødelighetsmodeller generelt og Lee-Carter modeller spesielt. Merk deg om det gjelder Lee-Carter modeller også uten cohort-effekter. Her kan du referere mye til \textcite{HUNT_Villegas_2015}, som snakker om Rubustness og Convergence. Ha dette i bakhodet i videre diskusjon. 
