\newpage
\section{Mortality Modeling as a Statistical Field}
\label{sec:literature_review_by_topic}
\textcolor{myDarkGreen}{Her kan du snakke litt om historien til mortality modellering, og gjennomgå de viktigste modellene. Du kan nevne Lee-Carter modellen, og hvordan mange forskjellige versjoner av den har vært populær. Du kan nevne alternative modeller, og si litt om hva som har blitt sagt og gjort med disse tidligere - gjerne sett i forhold til Lee-Carter. Du kan også nevne forskjeller i Cause-specific og generell mortality. For eksempel at cohort har vært vanlig i cause-specific mortality, og nå også blir vanligere for general mortality. }
Good statistical methods for modeling trends in mortality rates are important tools in several scientific fields. In the public sector, projections of general and cause-specific mortality are of key interest in, among others, the development of health policies and planning of social securities and pension funds, and in actuarial applications, mortality forecasts play an important role in the pricing of life insurances and pension products (\textcite{LeeCarter1992}, \textcite{Brouhns_2002}, \textcite{Czado_2005}, \textcite{Renshaw_Haberman_2006},\textcite{Renshaw_Haberman_2009}). The dominating approach to mortality modelling on a populational level is to consider mortality, it being observed counts of mortality, mortality rates or probability of mortality as a combination of age- and period specific effects. In some cases, a so-called cohort effect is included, which describes lifelong effects for the population with the same birth year. This general structure is common for most mortality models in literature. The differences between the numerous models that we see in the field of mortality modeling lies in the structure of the relationship between the time effects, how variability is captured by the model, whether a separate cohort effect is included and whether the model is considered in a Bayesian setting. \textcolor{myDarkGreen}{Dette har du egentlig ikke kilder på. Burde du ha det?}

\subsection{The Dominating Models for Mortality Modeling}
\label{sec:most-common-mortality-models}
\textcolor{myDarkGreen}{In this section, you will account for the modst popular mortality models. How are they structured, for what purpose are they mainly used. You do not need to talk about advantages/ disadvantages here. Just say what they are, how they are defined etc. Perhaps also mention common model choices for random effects. }

\textcolor{myDarkGreen}{The most basic mortality model is the APC model, which defines a linear, additive relationship between the age, period and cohort effects. This model has been used in medical and demographical sciences, and is often applied to cause-specific mortality \cite{RieblerThesis2010}. Its simple structure makes it easy to perform inference with, but also causes problems of identifiability, due to the linear relationship 
\begin{equation*}
    \text{cohort} = \text{period} - \text{age}.
\end{equation*}
}

In the actuarial sciences, the model proposed by \textcite{LeeCarter1992}, hereby referred to as the Lee-Carter model, has been popular ever since its introduction, and variations if this model has largely dominated the actuarial field (\textcite{Renshaw_Haberman_2009}, \textcite{booth_tickle_2008}). In its most basic form, the Lee-Carter model assumes a Gaussian distribution of the log-mortality rates, where the mean of the distribution is linked to some age- and period effects through a predictor:
\begin{equation}
    \log(m_{x,t}) = \alpha_x + \beta_x\kappa_t + \eps_{x,t}.
\end{equation}
Here $m_{x,t}$ are the central mortality rates and $\eps_{x,t}$ is an error term with zero mean and some variance, representing the variability in the mortality that cannot be explained by any age- or period effects. $\alpha_x$ and $\kappa_t$ are the main age- and period effects, and $\beta_x$ is an age-specific modulating factor, that says something about how sensitive the population at a given age is to period-specific changes in mortality. 

After its introduction, numerous extensions of the Lee-Carter model have been proposed. \textcite{Brouhns_2002} suggested to exchange the Gaussian log-mortality model structure with a Poisson distribution, considering the observed number of deaths to follow a Poisson distribution:
\begin{equation*}
    Y_{x,t} \sim \Poisson(E_{x,t}e^{\eta_{x,t}}), \quad \eta_{x,t} = \alpha_x + \beta_x\kappa_t.
\end{equation*}.
\textcite{Czado_2005} used this model structure in a Bayesian setting, and achieved convinsing results with this approach. \textcolor{myDarkGreen}{Usikker på dette..}
\textcolor{myDarkBlue}{SOMEONE: proposed to extend predictor of this Poisson-model structure with an additional error term to accound for overdispersion in the model. They argue that since the variability of the Poisson distribution is constrained/bounded? by the value of the mean, the plain Poisson model would not be able to correctly quantify all variability in the data. }

\subsubsection{Inclusion of a Cohort Term}
\label{sec:inclusion-of-cohort-term}
\textcolor{myDarkGreen}{This could be a subsection in its own - see how it is natural. But at least, talk about how some models include a cohort term and some does not. Talk about demographic significance? Talk about how the models are altered to inlclude the cohort term. }

\textcolor{myDarkBlue}{In the field of medical mortality modeling, the inclusion of the cohort effect have long been considered a useful tool to correctly capture mortality \cite{} ER DET FRA RIEBLER?? In recent years however, the need to include a cohort effect to describe tendencies in general mortality has been pointed out also in the actuarial community FINN UT HVEM DU KAN SITERE, DET ER MANGE. }

\subsection{Choices of Constraints}
\label{sec:common-choices-of-constraints}
The mortality models are typically not identifiable in itself, and thus it is common practice to impose some constraints on the random effects to ensure identifiability.
\textcolor{myDarkGreen}{The constraints are arbitrary, and can in theory be any constraint that sufficiently reduces the degrees of freedom of the model. However, the demographic interpretation/significance of the random effect changes according to the choice of priors, and some constraints that ensure easy interpretation of the time effects are commonly seen throughout literature. Skriv om hva de forskjellige bruker, og hva slags betydning forskjellige constraints gir. }

One common choice of constraints to let the first element of the period effect be at zero and to apply and sum-to-unit constraints on the modulating age effects:
\begin{equation}
    \kappa_0 = 0, \quad \sum_x\beta_x = 1.
    \label{eq:constr_first_kappa_zero}
\end{equation}
These constraints are used by e.g. \textcite{Renshaw_Haberman_2011}, .... \textcite{Renshaw_Haberman_2011} also applied, to the models including a distinct cohort effect, the constraint
\begin{equation}
    \beta_x > 0 \forall x,
\end{equation}
to be precautious. \textcolor{myDarkGreen}{Har ikke helt forstått hva de konkluderer med? Er det nyttig eller ikke?}

\textcite{NOTE: Vær sikker på at du faktisk trenger constraints på cohort-effekt for å ha identifiability!!! \textcite{Renshaw_Haberman_2011} ser ikke ut til å bruke det (men de bruker en two-step procedure for å fikse på forholdet mellom cohort, period og age... Se hva Andrea har gjort! Ganske sikker på at \textcite{Wisniowski2015} har brukt sum-to-zero constraints på cohort-effekten... }

\subsection{Mortality Models in a Bayesian Framework}
\label{sec:mortality_models_bayesian}
Her kan du snakke spesifikt om Bayesianske versjoner av Mortality modeller. Kanskje ikke her du skal argumentere for fordelene ved Bayesianske modeller, du kan evt referere til det hvis det står noe om det i annen litteratur - ja , det kan du! Snakk om utfordringer andre har hatt ved Bayesisk modellering - nevne convergence issues? Snakk om hvilke metoder som har vært brukt for å løse disse modellene.  

\subsection{Common Choices of Prior Distributions}
\label{sec:common_choices_of_priors}
\textcolor{myDarkGreen}{You are not quite sure whether you should include this...}

\subsection{Identifiability in Mortality Models}
\label{sec:identifiability_mortality_models}
Nevn resultatet til \textcite{hunt_blake_2020} som sier at dersom $\beta_x$ er non-parametric, så er cohort-effekten identifiserbar. Her kan du trekke inn den klassiske APC-modellen som eksempel på en modell der $\beta_x$ er parametrisk, og dermed også ikke-identifiserbar. Nevn også resultatet om at AP-modeller er definerte dersom man har gitte constraints. Nevn at H6B (oppdater kilde!!!) foreslår noen constraints på cohort-parameteren for å sikre identifiability, men at dette endrer demografisk signifikans og bør gjøres med omhu. 

\textcite{Renshaw_Haberman_2006} avoids the issues arising from the relation
\begin{equation*}
    \text{cohort} = \text{period} - \text{age}
\end{equation*}
by adopting the two-stage fitting procedure, first proposed by \textcite{LeeCarter1992}. This entails \textcolor{myDarkGreen}{Dobbeltsjekk} to first establish a static life table to represent the overall age effect $\alpha_x$, and then, conditional on this age effect, establish the period and cohort effects, $\kappa_t$ and $\gamma_{t-x}$ and the modulating age effects $\beta_x$. 

\subsection{Projection and Forecasting of Mortality}
\label{sec:forecasting_mortality}
Her må du sette deg godt inn i ting, for du synes at det er ganske vanskelig. 

Hvilke metoder har andre brukt for å projektere (forecast - forsøk å forstå begrepene godt) mortality rates. Hvilke utfordringer er knyttet til forskjellige metoder?

Gjennomgå - hvordan blir det gjort i Bayesiansk og frekventistisk sammenheng?

\subsection{Error Terms - Handling Uncertainty and Volality}
\label{sec:error_terms}
Si noe om hvordan utviklingen har gått. Fra enkel error term, til Poisson-antagelse, til dobbel-usikkerhet - med både error term og Poisson. Hvilke papers bruker de forskjellige (hvis det gir mening, kan være det blir litt oppramsende)

\subsection{Convergence in Mortality Models}
\label{sec:convergence_in_mortality_models}
\textcolor{myDarkGreen}{Not quite sure if you need to inlcude this, if it is not relevant for your thesis. }
Her kan du diskutere det som har blitt sagt om konvergensproblemer for dødelighetsmodeller generelt og Lee-Carter modeller spesielt. Merk deg om det gjelder Lee-Carter modeller også uten cohort-effekter. Her kan du referere mye til \textcite{HUNT_Villegas_2015}, som snakker om Rubustness og Convergence. Ha dette i bakhodet i videre diskusjon. 

The following papers models assumes the number of deaths to be Poisson distributed:
\begin{itemize}
    \item \textcite{Czado_2005}
    \item \textcite{Cairns_2009}
    \item \textcite{Wong_Forster_2018}
    \item \textcite{HUNT_Villegas_2015}
\end{itemize}

The following papers follow a Bayesian approach:
\begin{itemize}
    \item \textcite{Hunt_blake_2021}
    \item \textcite{Wong_Forster_2018}
\end{itemize}

% I still need to read through and consider the relevance of the following papers:
% \begin{itemize}
%     \item \textcite{Cairns_2009}
% \end{itemize}