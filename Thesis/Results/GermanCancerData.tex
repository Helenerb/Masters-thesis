\newpage
\subsection{German Cancer Data}
\label{sec:german_cancer_data}
\textcolor{myDarkRed}{NOTE: Are these central or initial exposures? By January 1st or something else? I think that might be important (at least it is a distinction other papers make), and I think it should be initial. At least, make sure that you have what you say that you have... They are actually "final" exposures.. i.e. exposures by the end of the year. Find out what to do about that... }
We use data of mortality for lung and stomach cancer in Germany for the years 1999-2016, obtained from \textcite{cancerData}. In these data sets, the observed mortality for the two cancer types is given by the number of cases for men and women, for each year, for five-year age intervals. The exception is for the ages above 85 year, which are collected in a single group. We denote the observed cases of deaths for one cancer type, for one sex, at age group $x$ and cohort $c$ during period $t$ by $Y_{x,t}^{\text{type, sex}}$. We note that the cohort $c$ of the observation is implicitly given by the age $x$ and the year $t$. 

We find data for the total German population, which serve as our at-risk population, for the corresponding years and ages at \textcite{germanPopulation}. This dataset contains observations for age groups of one year, and we aggregate them to the same five-year age groups as for the cancer data. We denote the observed German population for a given sex, for a given age group $x$ and cohort $c$ during period $t$ by $E_{x,t}^{\text{sex}}$. We then find the corresponding mortality rates by
\begin{equation}
    m_{x,t}^{\text{type, sex}} = \frac{Y_{x,t}^{\text{type, sex}}}{E_{x,t}^{\text{sex}}},
    \label{eq:expression_mr}
\end{equation}
or $Y_{x,t}^{\text{type}} = Y_{x,t}^{\text{type}}/E_{x,t}$ in the case where we consider the two sexes together.

When formatting the data sets, it is easier to begin the indexing of age groups, period and cohorts at 0 rather than 1, so we update the indexing scheme accordingly:
\begin{itemize}
    \item The age indices $x$ are now $x = 0,\ldots,X = 17$.
    \item The period indices $t$ are now $t = 0, \ldots, T = 17$.
    \item The cohort indices are now given by $c = 5(X - x) + t$, $C = 5X + T = 102$.
\end{itemize}
This way of indexing the cohort effects, as opposed to the canonical $c= t - x$, is applied because our data contains age- and period intervals of unequal lengths. This way of indexing cohort effects was proposed for APC models by \textcite{Heuer1997} and discussed by e.g. \textcite{RieblerThesis2010}. 

The summarized data is presented in Figures \ref{fig:obs-all-by-age}, \ref{fig:obs-all-by-period}, \ref{fig:obs-mr-by-age}, \ref{fig:obs-mr-by-period} and \ref{fig:obs-mr-by-cohort}. 

Figures \ref{fig:obs-all-by-age} and \ref{fig:obs-all-by-period} give representations of the distribution of the observed population and deaths over different age groups and years. From Figure \ref{fig:obs-all-by-age}, we see that there are most observed cancer deaths at around age group 70-75 ($x = 14$), for both male and females. From Figure \ref{fig:obs-all-total-by-age}, we can observe an aging in the German population, with the most common age shifting from around 40 years old in 1999 to closer to 50 years old in 2016.\textcolor{myDarkGreen}{This point was just made to say something about the plot, not really relevant to the research} From Figure \ref{fig:obs-all-by-period} we observe that the observed lung and stomach cancer cases seem to increase with time for women, but stay rather stable for men. We note that there is a clear drop in the German population in year 2011, which can be seen in Figure \ref{fig:obs-all-total-by-period}. This is an effect of the population census held in Germany in 2011, which showed that the estimated population numbers were off by 1.5 million people compared to the previous years \parencite{germanCensus}.
\begin{figure}
    \centering
    
    \begin{subfigure}[b]{.45\linewidth}
        \includegraphics[width=\linewidth]{Thesis/Results/Figures_RealData/dataPlotted/all_total_by_age.pdf}
        \label{fig:obs-all-total-by-age}
    \end{subfigure}
    \begin{subfigure}[b]{.45\linewidth}
        \includegraphics[width=\linewidth]{Thesis/Results/Figures_RealData/dataPlotted/all_stomach_by_age.pdf}
        \label{fig:obs-all-stomach-by-age}
    \end{subfigure}
    
    \begin{subfigure}[b]{.45\linewidth}
        \includegraphics[width=\linewidth]{Thesis/Results/Figures_RealData/dataPlotted/all_lung_by_age.pdf}
        \label{fig:obs-all-stomach-by-age}
    \end{subfigure}
    
    \caption{The total number of people (upper left) and stomach and lung cancer deaths (upper right and lower) are displayed by the different age groups, for the years 1999, 2007 and 2016.}
    \label{fig:obs-all-by-age}
\end{figure}

\begin{figure}
    \centering
    
    \begin{subfigure}[b]{.45\linewidth}
        \includegraphics[width=\linewidth]{Thesis/Results/Figures_RealData/dataPlotted/all_total_by_year.pdf}
        \label{fig:obs-all-total-by-period}
    \end{subfigure}
    \begin{subfigure}[b]{.45\linewidth}
        \includegraphics[width=\linewidth]{Thesis/Results/Figures_RealData/dataPlotted/all_stomach_by_year.pdf}
        \label{fig:obs-all-stomach-by-period}
    \end{subfigure}
    
    \begin{subfigure}[b]{.45\linewidth}
        \includegraphics[width=\linewidth]{Thesis/Results/Figures_RealData/dataPlotted/all_lung_by_year.pdf}
        \label{fig:obs-all-stomach-by-period}
    \end{subfigure}
    
    \caption{The total number of people (upper left) and stomach and lung cancer deaths (upper right and lower), summed over all age groups and displayed for each year.}
    \label{fig:obs-all-by-period}
\end{figure}
%Figure \ref{fig:data-rate} presents the observed mortality rates for years 1999, 2005, 2011 and 2016, for each age group and cohort. We observe that for younger ages, the mortality rates are close to zero for both lung and stomach cancer. For the years 2005, 2011 and 2016 we see, from Figure \ref{fig:data-rate-bottom} that we do not have observations of the  mortality rates for older cohorts (lower cohort index $k$). On the other side, we see that younger cohorts, at around $k > 60$, have mortality rates that are close to zero for all years. From Figure \ref{fig:data-rate-top} we see that the male and female mortality rates behave differently over the periods, for both lung and stomach cancer. As previously indicated, the female mortality rates are generally higher in 2016 compared to 1999, while the opposite is the case for male mortality. This difference is especially apparent for the lung cancer mortality. 

Figures \ref{fig:obs-mr-by-age} and \ref{fig:obs-mr-by-period} display the observed male, female and total mortality rates, as found from the relation in Equation \ref{eq:expression_mr}, averaged over all available years and age groups respectively. We first note that the male and female mortality rates behave quite differently to each other and to the mortality rates for the full population. This is the case for both lung and stomach cancer mortality, and we interpret this as an indication that male and female mortality should be treated separately in modeling. Our previous work on mortality modeling for this data confirms this \parencite{behrens_2021}.
% German cancer data - average over all years
\begin{figure}
    \centering
    \textbf{Mortality rates averaged over the period 1999-2016}
    
    \begin{subfigure}[b]{.45\linewidth}
        \includegraphics[width=\linewidth]{Thesis/Results/Figures_RealData/dataPlotted/mr_stomach_by_age.pdf}
        \caption{\textbf{Stomach}: Average mortality rates for stomach cancer, male and female}
        \label{fig:obs-mr-stomach-by-age}
    \end{subfigure}
    \begin{subfigure}[b]{.45\linewidth}
        \includegraphics[width=\linewidth]{Thesis/Results/Figures_RealData/dataPlotted/mr_lung_by_age.pdf}
        \caption{\textbf{Lung}: Average mortality rates for lung cancer, male and female}
        \label{fig:obs-mr-lung-by-age}
    \end{subfigure}
    \caption{Mortality rates averaged over all years, for male and females}
    \label{fig:obs-mr-by-age}
\end{figure}
% German cancer data - average over all ages
\begin{figure}
    \centering
    \textbf{Mortality rates averaged over ages 0-85+}
    
    \begin{subfigure}[b]{.45\linewidth}
        \includegraphics[width=\linewidth]{Thesis/Results/Figures_RealData/dataPlotted/mr_stomach_by_period.pdf}
        \caption{\textbf{Stomach}: Average mortality rates for stomach cancer, male and female}
        \label{fig:obs-mr-stomach-by-period}
    \end{subfigure}
    \begin{subfigure}[b]{.45\linewidth}
        \includegraphics[width=\linewidth]{Thesis/Results/Figures_RealData/dataPlotted/mr_lung_by_period.pdf}
        \caption{\textbf{Lung}: Average mortality rates for lung cancer, male and female}
        \label{fig:obs-mr-lung-by-period}
    \end{subfigure}
    \caption{Mortality rates averaged over all age groups, for male and females}
    \label{fig:obs-mr-by-period}
\end{figure}
Figure \ref{fig:obs-mr-by-cohort} displays the observed mortality rates for each cohort for the years 1999, 2007 and 2016. From this figure, we can see that we do not have observations for the older cohorts for all of the years. For instance, for the year 2016 we only have observations of the cohorts down to $c = 17$. This simply results from the fact that all observations older than 85 are counted together. This is of course a simplification of the reality. For example, there will be some small part of the population that were 85 years old or older in 1999, and thus part of the $c = 0$ cohort, that survived until 2016, where the $c=0$ cohort is no longer observed. In 2016, this same person would be assigned the cohort $c = 17$. Although we can technically consider this as an error in the cohort data, we assume that it will not affect our analysis significantly, since it applies to such as small part of the population. However, we note that we in general have fewer observations for the older cohorts. Similarly, we also have fewer observations for the cohorts born after 1999. Additionally, we see from Figure \ref{fig:obs-mr-by-cohort} that for many of the younger cohorts, we can only observe mortality rates close to zero during the entirety of the observed period 1999-2016. We keep this in mind when later considering the cohort estimates for these ages. 
% German cancer data  - average over all cohorts
\begin{figure}
    \centering
    \textbf{Mortality rates averaged over all birth years}
    
    \begin{subfigure}[b]{.45\linewidth}
        \includegraphics[width=\linewidth]{Thesis/Results/Figures_RealData/dataPlotted/mr_stomach_by_cohort.pdf}
        \caption{\textbf{Stomach}: Average mortality rates for stomach cancer, male and female}
        \label{fig:obs-mr-stomach-by-cohort}
    \end{subfigure}
    \begin{subfigure}[b]{.45\linewidth}
        \includegraphics[width=\linewidth]{Thesis/Results/Figures_RealData/dataPlotted/mr_lung_by_cohort.pdf}
        \caption{\textbf{Lung}: Average mortality rates for lung cancer, male and female}
        \label{fig:obs-mr-lung-by-cohort}
    \end{subfigure}
    \caption{Mortality rates averaged over all cohorts, for male and females}
    \label{fig:obs-mr-by-cohort}
\end{figure}