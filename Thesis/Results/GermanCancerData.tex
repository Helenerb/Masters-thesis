\newpage
\subsection{German Cancer Data}
\label{sec:german_cancer_data}
\textcolor{myDarkRed}{From project, revise!}
We use data of mortality for lung and stomach cancer in Germany for the years 1999-2016, obtained from \textcite{cancerData}. In these data sets, the observed mortality for the two cancer types is given by the number of cases for men and women, for each year, for five-year age intervals. The exception is for the ages above 85 year, which are collected in a single group. We denote the observed cases of deaths for one cancer type, for one sex, at age group $x$ and cohort $c$ during period $t$ by $Y_{x,t}^{\text{type, sex}}$. \textcolor{myDarkBlue}{We note that the cohort $c$ is implicitly given by the age group $x$ and the period $t$.} For the univariate part of our analysis, we add the male and female cases to obtain the total mortality $Y_{x,t}^{\text{type}}$. We find data for the total German population for the corresponding years and ages at \textcite{germanPopulation}. This dataset contains observations for age groups of one year, and we aggregate them to the same five-year age groups as for the cancer data. We denote the observed German population for a given sex, for a given age group $x$ and cohort $c$ during period $t$ by $E_{x,t}^{\text{sex}}$. We then find the corresponding mortality rates by
\begin{equation}
    m_{x,t}^{\text{type, sex}} = \frac{C_{x,t}^{\text{type, sex}}}{P_{x,t}^{\text{sex}}},
\end{equation}
or $Y_{x,t}^{\text{type}} = C_{x,t}^{\text{type}}/P_{x,t}$ in the case where we consider the two sexes together.

\newpar When formatting the data sets, it is easier to begin the indexing of age groups, period and cohorts at 0 rather than 1, so we update the indexing scheme accordingly:
\begin{itemize}
    \item The age indices $x$ are now $x = 0,\ldots,X$.
    \item The period indices $t$ are now $t = 0, \ldots, T$.
    \item The cohort indices are now given by $c = 5(X - x) + t$, $C = 5X + T$.
\end{itemize}
\textcolor{myDarkBlue}{We index the cohort effects like this, instead of the canonical $c = t - x$, because we treat data where the period- and age intervals are of different lengths. }

\newpar A summary of the data sets is presented in Figures \ref{fig:data-total} and \ref{fig:data-rate}. Figure \ref{fig:data-total} gives a representation of the distribution of the population and the distribution of mortality cases over different age groups and years. From Figure \ref{fig:data-total-top}, we see that there are most observed cancer deaths at around age group 70-75 ($x = 14$), for both male and females. From Figure \ref{fig:data-total-bottom} we observe that the observed lung and stomach cancer cases seem to increase with time for women, but stay rather stable for men. We note that there is a clear drop in the German population in year 2011. This is because Germany held a population census in 2011, which showed that the estimated population numbers from the previous years were off by 1.5 million people \parencite{germanCensus}.

\begin{figure}
    \centering
    \begin{subfigure}[b]{.75\linewidth}
        \includegraphics[width=\linewidth]{real-data/real-data-univariate/Figures/data-age-total.png}
        \caption{The total number of people (left) and stomach and lung cancer deaths (middle and right) are displayed by the different age groups, for the years 1999, 2007 and 2016.}
        \label{fig:data-total-top}
    \end{subfigure}
    
    \begin{subfigure}[b]{.75\linewidth}
        \includegraphics[width=\linewidth]{real-data/real-data-univariate/Figures/data-year-total.png}
        \caption{The total number of people (left) and stomach and lung cancer deaths (middle and right) are summed over all age groups, and displayed by the calendar years. }
        \label{fig:data-total-bottom}
    \end{subfigure}
    \caption{The total number of people in the German population and the total observed cases for lung and stomach cancer deaths.}
    \label{fig:data-total}
\end{figure}

\newpar Figure \ref{fig:data-rate} presents the observed mortality rates for years 1999, 2005, 2011 and 2016, for each age group and cohort. We observe that for younger ages, the mortality rates are close to zero for both lung and stomach cancer. For the years 2005, 2011 and 2016 we see, from Figure \ref{fig:data-rate-bottom} that we do not have observations of the  mortality rates for older cohorts (lower cohort index $k$). On the other side, we see that younger cohorts, at around $k > 60$, have mortality rates that are close to zero for all years. From Figure \ref{fig:data-rate-top} we see that the male and female mortality rates behave differently over the periods, for both lung and stomach cancer. As previously indicated, the female mortality rates are generally higher in 2016 compared to 1999, while the opposite is the case for male mortality. This difference is especially apparent for the lung cancer mortality. 

\begin{figure}
    \centering
    \begin{subfigure}[b]{.6\linewidth}
        \includegraphics[width=\linewidth]{real-data/real-data-univariate/Figures/data-age-rate.png}
        \caption{The observed mortality rates, displayed by the age groups.}
        \label{fig:data-rate-top}
    \end{subfigure}
    
    \begin{subfigure}[b]{.6\linewidth}
        \includegraphics[width=\linewidth]{real-data/real-data-univariate/Figures/data-cohort-rate.png}
        \caption{The observed mortality rates, displayed by the cohorts.}
        \label{fig:data-rate-bottom}
    \end{subfigure}
    \caption{The observed mortality rates for the German lung (left) and stomach (right) cancer data, in the years 1999, 2005, 2011 and 2016.}
    \label{fig:data-rate}
\end{figure}

\texcolor{myDarkRed}{Specific to masters:}
Figures \ref{fig:obs-mr-by-age} and \ref{fig:obs-mr-by-period} displays the observed mortality rates for males and females, averaged over all years 1999-2016 and all ages 0-85+ respectively. Figure \ref{fig:obs-mr-by-cohort} display the observed mortality rates for male...
\textcolor{myDarkGreen}{NOTE: Are these central or initial exposures? By January 1st or something else? I think that might be important (at least it is a distinction other papers make), and I think it should be initial. At least, make sure that you have what you say that you have...}

% German cancer data - average over all years
\begin{figure}
    \centering
    \textbf{Mortality rates averaged over the period 1999-2016}
    
    \begin{subfigure}[b]{.45\linewidth}
        \includegraphics[width=\linewidth]{Thesis/Results/Figures_RealData/dataPlotted/mr_stomach_by_age.pdf}
        \caption{\textbf{Stomach}: Average mortality rates for stomach cancer, male and female}
        \label{fig:obs-mr-stomach-by-age}
    \end{subfigure}
    \begin{subfigure}[b]{.45\linewidth}
        \includegraphics[width=\linewidth]{Thesis/Results/Figures_RealData/dataPlotted/mr_lung_by_age.pdf}
        \caption{\textbf{Lung}: Average mortality rates for lung cancer, male and female}
        \label{fig:obs-mr-lung-by-age}
    \end{subfigure}
    \caption{Mortality rates averaged over all years, for male and females}
    \label{fig:obs-mr-by-age}
\end{figure}

% German cancer data - average over all ages
\begin{figure}
    \centering
    \textbf{Mortality rates averaged over ages 0-85+}
    
    \begin{subfigure}[b]{.45\linewidth}
        \includegraphics[width=\linewidth]{Thesis/Results/Figures_RealData/dataPlotted/mr_stomach_by_period.pdf}
        \caption{\textbf{Stomach}: Average mortality rates for stomach cancer, male and female}
        \label{fig:obs-mr-stomach-by-period}
    \end{subfigure}
    \begin{subfigure}[b]{.45\linewidth}
        \includegraphics[width=\linewidth]{Thesis/Results/Figures_RealData/dataPlotted/mr_lung_by_period.pdf}
        \caption{\textbf{Lung}: Average mortality rates for lung cancer, male and female}
        \label{fig:obs-mr-lung-by-period}
    \end{subfigure}
    \caption{Mortality rates averaged over all age groups, for male and females}
    \label{fig:obs-mr-by-period}
\end{figure}

% German cancer data  - average over all cohorts
\begin{figure}
    \centering
    \textbf{Mortality rates averaged over all birth years}
    
    \begin{subfigure}[b]{.45\linewidth}
        \includegraphics[width=\linewidth]{Thesis/Results/Figures_RealData/dataPlotted/mr_stomach_by_cohort.pdf}
        \caption{\textbf{Stomach}: Average mortality rates for stomach cancer, male and female}
        \label{fig:obs-mr-stomach-by-cohort}
    \end{subfigure}
    \begin{subfigure}[b]{.45\linewidth}
        \includegraphics[width=\linewidth]{Thesis/Results/Figures_RealData/dataPlotted/mr_lung_by_cohort.pdf}
        \caption{\textbf{Lung}: Average mortality rates for lung cancer, male and female}
        \label{fig:obs-mr-lung-by-cohort}
    \end{subfigure}
    \caption{Mortality rates averaged over all cohorts, for male and females}
    \label{fig:obs-mr-by-cohort}
\end{figure}