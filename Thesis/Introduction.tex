\newpage
\section{Introduction}

We will investigate if it is possible to use the newly proposed \texttt{R}-package \inlabru to make fast Bayesian inference on Lee-Carter types of models. We will specifically consider some versions of the Lee-Carter model applied to lung and stomach cancer data, but we note that our findings could be generalized to other models with a Lee-Carter type of structure, as well as general mortality in addition to cause-specific mortality. 

\textcolor{myDarkGreen}{On Motivation of Thesis:}
\textcolor{myDarkGreen}{Remember: even if your exact model formulation is not identifiable, you can still show that it is possible to use \inlabru with the Lee-Carter model structure. This is still useful! So you can say that we consider an easy version of the model, but perhaps that this seem to have some identifiability issues bla bla.}