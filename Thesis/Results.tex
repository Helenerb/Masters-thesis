\section{Results}
First, we use synthetic data to show that $\texttt{inlabru}$ produces restults that are sufficiently similar to the one produced by more traditional MCMC approaches. For a fair comparison between the $\texttt{INLA}$ framework and state-of-the-art MCMC methods, we compare results of Bayesian inference using $\texttt{inlabru}$ to results using inference from the STAN methodology. The STAN methodology is an implementation of a Hamiltonian Monte Carlo approach, and is described more closely in Section \ref{sec:stan}. We use the $\texttt{R}$-library $\texttt{rstan}$ to do inference with STAN. 

\subsection{STAN: Hamiltonian Monte Carlo}
\label{sec:stan}
STAN is a method for Bayesian inference using a Hamiltonian Monte Carlo approach. It was proposed by ... and has since its introduction gained popularity for its computational power compared to traditional MCMC approaches. bla bla ikke ferdig!!!

\subsection{Preliminary Analysis With Synthetic Data}
\label{sec:synthetic-data}


\subsection{Applying $\texttt{inlabru}$ to German Cancer Data}
\label{sec:real-data}

In this second part of our analysis, we will apply the previously described approach for Bayesian analysis with the Lee-Carter (LC) and the cohort-extended Lee-Carter (LCC) model to cancer data. We will use data of German mortality of lung and stomach cancer data for the years 1999-2000. 

\begin{figure}
    \centering
    \includegraphics[width=0.85\linewidth]{does/not/exist}
    \caption{Template of single figure}
    \label{fig:uv-full-data-LC-l}
\end{figure}

\begin{figure}
    \centering
    \begin{subfigure}[b]{.6\linewidth}
        \includegraphics[width=\linewidth]{does/not/exist}
        \caption{Subcaption}
        \label{fig:data-rate-top}
    \end{subfigure}
    
    \begin{subfigure}[b]{.6\linewidth}
        \includegraphics[width=\linewidth]{does/not/exist}
        \caption{Subcaption}
        \label{fig:data-rate-bottom}
    \end{subfigure}
    \caption{Template of double figure}
    \label{fig:data-rate}
\end{figure}

\begin{figure}
    \centering
    \begin{subfigure}[b]{.6\linewidth}
        \includegraphics[width=\linewidth]{does/not/exist}
        \caption{Subcaption}
        \label{fig:data-rate-top}
    \end{subfigure}
    
    \begin{subfigure}[b]{.6\linewidth}
        \includegraphics[width=\linewidth]{does/not/exist}
        \caption{Subcaption}
        \label{fig:data-rate-bottom}
    \end{subfigure}
    \caption{Template of double figure}
    \label{fig:data-rate}
\end{figure}

