\section{Results}
First, we use synthetic data to show that $\texttt{inlabru}$ produces restults that are sufficiently similar to the one produced by more traditional MCMC approaches. For a fair comparison between the $\texttt{INLA}$ framework and state-of-the-art MCMC methods, we compare results of Bayesian inference using $\texttt{inlabru}$ to results using inference from the STAN methodology. The STAN methodology is an implementation of a Hamiltonian Monte Carlo approach, and is described more closely in Section \ref{sec:stan}. We use the $\texttt{R}$-library $\texttt{rstan}$ to do inference with STAN. 

\subsection{STAN: Hamiltonian Monte Carlo}
\label{sec:stan}
STAN is a method for Bayesian inference using a Hamiltonian Monte Carlo approach. It was proposed by ... and has since its introduction gained popularity for its computational power compared to traditional MCMC approaches. bla bla ikke ferdig!!!

\subsection{Preliminary Analysis With Synthetic Data}
\label{sec:synthetic-data}


\subsection{Applying $\texttt{inlabru}$ to German Cancer Data}
\label{sec:real-data}

In this second part of our analysis, we will apply the previously described approach for Bayesian analysis with the Lee-Carter (LC) and the cohort-extended Lee-Carter (LCC) model to cancer data. We will use data of German mortality of lung and stomach cancer data for the years 1999-2000. 

Figures \ref{fig:mr-lcc-stomach} and \ref{fig:random-effects-lcc-stomach} displays the stomach cancer mortality rates and age, period and cohort effects, with their associated hyperparameters, as estimated by $\texttt{inlabru}$ for the LDD-model. These results are produced in the script \url{Master Thesis Code/Scripts/Real data/run_inlabru_real_data.R}.

\begin{figure}
    \centering
    \begin{subfigure}[b]{.45\linewidth}
        \includegraphics[width=\linewidth]{Master Thesis Code/Scripts/Real data/Output/Figures/stomach_rw2/mr_x_inlabru.pdf}
        \caption{Mortality rates displayed as a function of calendar year, for each age group. }
        \label{fig:mr-lcc-stomach-x}
    \end{subfigure}
    \begin{subfigure}[b]{.45\linewidth}
        \includegraphics[width=\linewidth]{Master Thesis Code/Scripts/Real data/Output/Figures/stomach_rw2/mr_t_inlabru.pdf}
        \caption{The mortality rates displayed as a function of age, for each available calendar year. }
        \label{fig:mr-lcc-stomach-t}
    \end{subfigure}
    \caption{The stomach cancer mortality rate estimated by $\texttt{inlabru}$ displayed together with the observed stomach cancer mortality rate.}
    \label{fig:mr-lcc-stomach}
\end{figure}

\begin{figure}
    \centering
    \begin{subfigure}[b]{.85\linewidth}
        \includegraphics[width=\linewidth]{Master Thesis Code/Scripts/Real data/Output/Figures/stomach_rw2/random_effects_inlabru.pdf}
        \caption{Estimated random effects.}
        \label{fig:random-effects-lcc-stomach-re}
    \end{subfigure}
    
    \begin{subfigure}[b]{.85\linewidth}
        \includegraphics[width=\linewidth]{Master Thesis Code/Scripts/Real data/Output/Figures/stomach_rw2/hypers_inlabru.pdf}
        \caption{Estimated hyperparameters}
        \label{fig:random-effects-lcc-stomach-hyper}
    \end{subfigure}
    \caption{The random age- period and cohort effects, as well as the hyperparameters, estimated by applying the LCC-model to stomach cancer mortality data by $\texttt{inlabru}$.}
    \label{fig:random-effects-lcc-stomach}
\end{figure}

Figures \ref{fig:mr-lcc-lung} and \ref{fig:random-effects-lcc-lung} displays the lung cancer mortality rates and age, period and cohort effects, with their associated hyperparameters, as estimated by $\texttt{inlabru}$ for the LDD-model. These results are produced in the script \url{Master Thesis Code/Scripts/Real data/run_inlabru_real_data.R}.

\begin{figure}
    \centering
    \begin{subfigure}[b]{.45\linewidth}
        \includegraphics[width=\linewidth]{Master Thesis Code/Scripts/Real data/Output/Figures/lung_rw2/mr_x_inlabru.pdf}
        \caption{Mortality rates displayed as a function of calendar year, for each age group. }
        \label{fig:mr-lcc-lung-x}
    \end{subfigure}
    \begin{subfigure}[b]{.45\linewidth}
        \includegraphics[width=\linewidth]{Master Thesis Code/Scripts/Real data/Output/Figures/lung_rw2/mr_t_inlabru.pdf}
        \caption{The mortality rates displayed as a function of age, for each available calendar year. }
        \label{fig:mr-lcc-lung-t}
    \end{subfigure}
    \caption{The lung cancer mortality rate estimated by $\texttt{inlabru}$ displayed together with the observed lung cancer mortality rate.}
    \label{fig:mr-lcc-lung}
\end{figure}

\begin{figure}
    \centering
    \begin{subfigure}[b]{.85\linewidth}
        \includegraphics[width=\linewidth]{Master Thesis Code/Scripts/Real data/Output/Figures/lung_rw2/random_effects_inlabru.pdf}
        \caption{Estimated random effects.}
        \label{fig:random-effects-lcc-lung-re}
    \end{subfigure}
    
    \begin{subfigure}[b]{.85\linewidth}
        \includegraphics[width=\linewidth]{Master Thesis Code/Scripts/Real data/Output/Figures/lung_rw2/hypers_inlabru.pdf}
        \caption{Estimated hyperparameters}
        \label{fig:random-effects-lcc-lung-hyper}
    \end{subfigure}
    \caption{The age- period and cohort effects, as well as the hyperparameters, estimated by applying the LCC-model to lung cancer mortality data by $\texttt{inlabru}$.}
    \label{fig:random-effects-lcc-lung}
\end{figure}

Figures \ref{fig:mr-pred-stomach} and \ref{fig:random-effects-pred-stomach} displays the results of applying the LCC-model to German stomach cancer mortality data. The observed mortality rates for years 1999-2007 are used to predict the mortality rates, and the associated age, period and cohort effects, for the years 2008-2016. These results are produced in the script \url{Master Thesis Code/Scripts/Real data/run_inlabru_real_data.R}.

\begin{figure}
    \centering
    \begin{subfigure}[b]{.85\linewidth}
        \includegraphics[width=\linewidth]{Master Thesis Code/Scripts/Real data/Output/Figures/stomach_rw2_predict/mr_x_inlabru.pdf}
        \caption{Mortality rates displayed as a function of calendar year, for each age group. }
        \label{fig:mr-pred-stomach-x}
    \end{subfigure}
    
    \begin{subfigure}[b]{.85\linewidth}
        \includegraphics[width=\linewidth]{Master Thesis Code/Scripts/Real data/Output/Figures/stomach_rw2_predict/mr_t_inlabru.pdf}
        \caption{The mortality rates displayed as a function of age, for each available calendar year. }
        \label{fig:mr-pred-stomach-t}
    \end{subfigure}
    \caption{The stomach cancer mortality rate estimated and predicted by $\texttt{inlabru}$ displayed together with the observed stomach cancer mortality rate. The predicted mortality rates are marked with blue triangles, while the estimated mortality rates for which we have observed data are marked with blue crosses.}
    \label{fig:mr-pred-stomach}
\end{figure}

\begin{figure}
    \centering
    \begin{subfigure}[b]{.85\linewidth}
        \includegraphics[width=\linewidth]{Master Thesis Code/Scripts/Real data/Output/Figures/stomach_rw2_predict/random_effects_inlabru.pdf}
        \caption{Estimated random effects. The estimates for the period effect $\kappa_t$ for the unobserved years are marked with yellow triangles. The estimates for the cohort effect $\gamma_c$ for the partially observed and unobserved cohorts are marked with green crosses and yellow triangles, respectively. }
        \label{fig:random-effects-pred-stomach-re}
    \end{subfigure}
    
    \begin{subfigure}[b]{.85\linewidth}
        \includegraphics[width=\linewidth]{Master Thesis Code/Scripts/Real data/Output/Figures/stomach_rw2_predict/hypers_inlabru.pdf}
        \caption{Estimated hyperparameters}
        \label{fig:random-effects-pred-stomach-hyper}
    \end{subfigure}
    \caption{The random age- period and cohort effects, as well as the hyperparameters, estimated by applying the LCC-model to stomach cancer mortality data by $\texttt{inlabru}$.}
    \label{fig:random-effects-pred-stomach}
\end{figure}

Figures \ref{fig:mr-pred-lung} and \ref{fig:random-effects-pred-lung} displays the results of applying the LCC-model to German stomach cancer mortality data. The observed mortality rates for years 1999-2007 are used to predict the mortality rates, and the associated age, period and cohort effects, for the years 2008-2016. 


\begin{figure}
    \centering
    \begin{subfigure}[b]{.85\linewidth}
        \includegraphics[width=\linewidth]{Master Thesis Code/Scripts/Real data/Output/Figures/lung_rw2_predict/mr_x_inlabru.pdf}
        \caption{Mortality rates displayed as a function of calendar year, for each age group. }
        \label{fig:mr-pred-lung-x}
    \end{subfigure}
    
    \begin{subfigure}[b]{.85\linewidth}
        \includegraphics[width=\linewidth]{Master Thesis Code/Scripts/Real data/Output/Figures/lung_rw2_predict/mr_t_inlabru.pdf}
        \caption{The mortality rates displayed as a function of age, for each available calendar year. }
        \label{fig:mr-pred-lung-t}
    \end{subfigure}
    \caption{The lung cancer mortality rate estimated and predicted by $\texttt{inlabru}$ displayed together with the observed lung cancer mortality rate. The predicted mortality rates are marked with blue triangles, while the estimated mortality rates for which we have observed data are marked with blue crosses.}
    \label{fig:mr-pred-lung}
\end{figure}

\begin{figure}
    \centering
    \begin{subfigure}[b]{.85\linewidth}
        \includegraphics[width=\linewidth]{Master Thesis Code/Scripts/Real data/Output/Figures/lung_rw2_predict/random_effects_inlabru.pdf}
        \caption{Estimated random effects. The estimates for the period effect $\kappa_t$ for the unobserved years are marked with yellow triangles. The estimates for the cohort effect $\gamma_c$ for the partially observed and unobserved cohorts are marked with green crosses and yellow triangles, respectively. }
        \label{fig:random-effects-pred-lung-re}
    \end{subfigure}
    
    \begin{subfigure}[b]{.85\linewidth}
        \includegraphics[width=\linewidth]{Master Thesis Code/Scripts/Real data/Output/Figures/lung_rw2_predict/hypers_inlabru.pdf}
        \caption{Estimated hyperparameters}
        \label{fig:random-effects-pred-lung-hyper}
    \end{subfigure}
    \caption{The random age- period and cohort effects, as well as the hyperparameters, estimated by applying the LCC-model to lung cancer mortality data by $\texttt{inlabru}$.}
    \label{fig:random-effects-pred-lung}
\end{figure}

